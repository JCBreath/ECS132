\documentclass{amsart}

\usepackage{enumitem}
\setlist[1]{itemsep=4ex}
\setlist[2]{itemsep=1ex}
\renewcommand\lnot{\sim}

\begin{document}
\title{Homework assignment 1}
\author{Siyuan Yao, Kai Yan, Zhengfei Chen}
\maketitle

\thispagestyle{empty}
\pagestyle{empty}

\begin{enumerate}
\item Problem A\\\\
We know that\\\\$ \dfrac{P(S=j|T=i) = P(S=j \text{ and } T=i)}{ P(T=i)}$\\ and also, if S and T are independet, this becomes\\ $P(S=j|T=i) = \dfrac{P(S=j)\cdot P(T=i)}{P(T=i)} = P(S=j)$\\
However, if we consider the case where j = 2 and T = 0, we can see that they are not independent.\\
P(S=2) means that both dices roll to 1,\\which has the possibility of $\dfrac{1}{6}\cdot\dfrac{1}{6} = \dfrac{1}{36}$.\\
P(T=0) means that both dices roll to odd number, which has the possibility of $\dfrac{1}{2}\cdot\dfrac{1}{2} = \dfrac{1}{4}$.\\
However, $P(S=2|T=0) = \dfrac{P(S=2 \text{ and } T=0)}{P(T=0)}$: because for all S=2, only rolling two dices that both have value 1 will hold, \\so P(S=2 and T=0)=P(S=2);\\
So $P(S=2|T=0) = \dfrac{P(S=2)}{ P(T=0)} = \dfrac{\dfrac{1}{36}}{\dfrac{1}{4}}$, which is certainly not the same as P(S=2), which is $\dfrac{1}{36}$.\\
So S and T are not independent.
\item Problem B\\
\begin{enumerate}
\item Expected Value and Variance of the degree of $v_1$\\
Let $D_n$ denotes the degree of node $N$ at the time immediately after $v_4$ is added\\
$P(D_1=3) = P(N_3=1 \text{ and } N_4=1) = P(N_4=1 | N_3=1)\cdot P(N_3=1) = 2/4 \cdot 1/2 = 1/4$\\
$P(D_1=2) = P(N_3=1 \text{ and } N_4=2) + P(N_3=1 \text{ and } N_4=3) + P(N_3=2 \text{ and } N_4=1) = 1/2 \cdot 1/4 \cdot 3 = 3/8$\\
$P(D_1=1) = P(N_3=2 \text{ and } N_4=2) + P(N_3=2 \text{ and } N_4=3) = 1/4 + 1/8 = 3/8$\\\\

$E(D_1)=3\cdot P(D_1=3)+2\cdot P(D_1=2)+1\cdot (D_1=1) = 3/4 + 6/8 + 3/8 = 15/8 = 1.875$\\\\
$Var(D_1)=E(D_1^2)-(ED_1)^2\\ = 3^2\cdot P(D_1=3)+2^2\cdot P(D_1=2)+1^2\cdot (D_1=1) - (15/8)^2$\\
$=9/4+12/8+3/8-225/16=144/64+96/64+24/64-225/64=39/64$\\
$Var(D_1)=39/64$\\

\item Covariance between the degrees of $v_1$ and $v_2$. \\
$E(D_2)=E(D_1)=1.875$ because ($P(N_3=1)=P(N_3=2)$)\\
$Cov(D_1,D_2)=E[(D_1-ED_1)(D_2-ED_2)]=(3-15/8)(1-15/8)\cdot 1/4+(2-15/8)(1-15/8)\cdot 1/8+(2-15/8)(2-15/8)\cdot2/8+(1-15/8)(2-15/8)\cdot 1/8+(1-15/8)(3-15/8)\cdot 2/8=-33/64$\\



\end{enumerate}

\item Problem C \\
$\mu = EB = \displaystyle \sum_{c=0}^3 c\cdot P(X=c) = 0\times0.5 + 1\times0.4 + 2\times0.1 = 0.6$\\
$Var(B) = (0 - 0.6)^2 \cdot 0.5 + (1-0.6)^2 \cdot0.4 + (2-0.6)^2 \cdot0.1 = 0.44$\\
$\sigma = \text{Standard Deviation of B} = \sqrt{Var(B)}= 0.66$\\
$E[(B-\mu)^3/\sigma^3]= \dfrac{(0-0.6)^3\cdot0.5+(1-0.6)^3\cdot0.4+(2-0.6)^3\cdot0.1}{0.66^3} = 0.658$\\




\item Problem D \\

$EB =\displaystyle \sum_{c=1}^{6} c \cdot \frac{1}{6} \cdot \frac{1}{6}$\\
$=\displaystyle \frac{1}{36}*\sum_{c=1}^{6} c $\\
 $   =\displaystyle \frac{1}{36} * 21 = 0.583 $\\
    
$E(B)^2 =\displaystyle \sum_{c=1}^{6} c^2 \cdot \frac{1}{6} \cdot \frac{1}{6} $\\
$=\displaystyle \frac{1}{36}\cdot\sum_{c=1}^{6} c^2$\\
$=\displaystyle \frac{1}{36} \cdot 91 = 2.528$\\

\end{enumerate}

\end{document}
