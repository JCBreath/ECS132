\documentclass{amsart}
\usepackage{tikz}
\usepackage{pgfplots}
\usepackage{amsmath}
\usepackage{enumitem}
\usepackage{mathtools}
\setlist[1]{itemsep=3ex}
\setlist[2]{itemsep=1ex}
\newcommand\tab[1][1cm]{\hspace*{#1}}

\begin{document}
\title{Homework assignment 4}
\author{Siyuan Yao, Zhengfei Chen, Kai Yan}
\maketitle

\thispagestyle{empty}
\pagestyle{empty}

\begin{enumerate}
\item Problem A\\
\begin{enumerate}
\item Since ducd wants to find the density at the value x, it with solved original function $f_x(x)$ with given value:\\
$ducd(x,c)=\dfrac{3}{2\cdot(c^{1.5}-1)}\cdot x^{0.5}$\\
\item pucd wants to find the cdf at value in q, which means it want to find the Fx(x), which is the integral of the fx(x).\\
$pucd(q,c)=\displaystyle\int^{q}_{1} \dfrac{3}{2\cdot(c^{1.5}-1)}\cdot x^{0.5} dx=\dfrac{q^{1.5}}{c^{1.5}-1}-\dfrac{1^{1.5}}{c^{1.5}-1}=\dfrac{q^{1.5}-1}{c^{1.5}-1}$\\
\item qucd wants to find the quantiles at the values of q. This means it will get want to find the inverse of cdf function.\\
$qucd(q,c)=(q\cdot(c^{1.5}-1)+1)^{\frac{2}{3}}$\\
\item In rucd, we want to generate n random values. So, we use runif to generate n random probabilities and use qucd to get the value of t's that have those probabilities.\\
rucd $<$- function(n,c) \{\\
\tab tmp $<$- runif(n);\\
\tab qucd(tmp,c);\\
\};\\
\end{enumerate}
\newpage
\item Extra Credit
\begin{enumerate}
\item Expected value\\\\
According to the given formula:\\
$ EV = E[g(U)]$\\
where \\$g(t) = E(V | U = t)$\\
we plug in \\$V = D $\\and \\$U = (N,p)$\\
and get:\\
\\$ED = E(E(D|(N,p)=t))$\\\\
since " Given N and p, D has a binomial distribution with N trials and success probability p",we have:\\
\\$ED = E(N*p)$\\\\
since N and p are independent as stated in the prompt\\
\\$ED = E(N*p)=E(N)*E(p)$\\\\
The expected value for Poisson distribution is $\lambda$\ , while the expected value for Beta distribution is $\alpha / (\alpha + \beta)$\\
so we get:\\
\\$ED = E(N*p)=E(N)*E(p)=\lambda*\alpha / (\alpha + \beta)$
\item Variance\\\\
again,according to the given formula:\\
$Var(V) = E[v(U)] + Var[e(U)]$\\
where $v(t) = Var(V | U = t)$ and $e(t) = E(V | U = t) $\\
from that we can derive:\\
$Var(V)= E[v(U)] + E[(e(U))^2] + (E[e(U)])^2$\\
again,we plug in \\$V = D $\\and \\$U = (N,p)$\\
so:\\
$Var(D)= E[v(D)] + E[(e(D))^2] + (E[e(D)])^2$\\
$=\int_{}^{} N *fu(N) * p * fu(p) * (1 - p * fu(p))dNdp$\\
$+\int_{}^{} (N *fu(N) * p * fu(p))^2 )dNdp$\\
$-\int_{}^{} (N *fu(N) * p * fu(p)) )dNdp$\\
we plug in:\\
$\int_{}^{}((p*fu(p))^2)dp = E(p^2) = Var(p) + (E(p))^2 = (\alpha+(\alpha)^2*(\alpha+\beta+1)/((\alpha+\beta)^2)*(\alpha+\beta+1))$\\
and:\\
$\int_{}^{}(N*fu(N))^2dN = E(N^2)=Var(N)+(E(N))^2=\lambda + (\lambda)^2$\\
then we can integral everything and get:
$Var(D)= $\\
\\$((\lambda)^2*(\alpha*\lambda+(\alpha)^2*(\alpha+\lambda+1))/((\alpha+\lambda)^2*(\alpha+\lambda+1)))+(\lambda*a/(\alpha+\beta))-(\lambda*\alpha/(\alpha+\beta))$\\

\end{enumerate}
\end{enumerate}
\end{document}
